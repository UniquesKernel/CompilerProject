\section{Conclusion}
\label{sec:conclusion}

In response to the growing popularity of interpreted, dynamically and weakly typed programming
languages such as \texttt{Python} and \texttt{JavaScript}, the intension was to offer
a compiled alternative, called \lang, with a clear explicit syntax, that is both statically and strongly
typed and offers type safety and memory safety at compile time without the overhead
of a garbage collector. \\ 

\lang{} offers many of the same constructs as modern programming languages,
such as immutable variables, basic arithmetic, functions, if-else constructs and
support for terminal outputs. \\

The project succeeds in the implementation of a \lexer{}, \parser{}, 
\typeChecker{}, and a \codeGen{} that allows us to fully parse a \lang{} program,
generating an \ast{} and ensuring type consistency throughout the \ast{}. This is
done by
Implmenting a \typeChecker{} using a visitor pattern that also allows the seemless
implementation of a \borrowChecker{} to enforce the same ownership and borrowing
rules as \texttt{Rust}. Lastly the implementation of a \codeGen{} using LLVM and the
\gcc{} allows us to seemlessly generate executable
files for any architecture. \\

The project encountered challenges in fully implementing the Rust-like
ownership and borrowing rules.
The \borrowChecker{} managed ownership within functions but faced difficulties with
ownership transfer between functions. Consequently this aspect of the project did
not fully meet the objectives set forth in sections~\ref{sec:Objectives} and
\ref{sec:Requirements}. \\

The use of a flexible iterative "minimal viable progress" development process allowed
the group to explore the different aspects of the problem, and face the challenges
inherent in such projects. While the group didn't manage to solve all the objectives
of the project, the process proved well suited for this kind task and for the group
size and would likely serve well in the develop of future language features.

\newpage
