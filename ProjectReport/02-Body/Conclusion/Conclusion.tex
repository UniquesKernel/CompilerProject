\section{Conclusion}
\label{sec:conclusion}

The \lang{} language has evolved to support common data types such as integers,
floating-point numbers, booleans, and characters. These can be stored in variables
for subsequent use. \lang{} supports essential features like arithmetic operations,
branching, and loop creation via recursive function calls. It does fall short on the
ability to take inputs from users as it is limited to only producing outputs. \\

A significant emphasis was placed on explicitness in coding. This was realized
through a static type system without type inference and by making variables
immutable by default. Mutable states are allowed only through a specialized keyword,
ensuring deliberate intentions for mutability. \\

However, the project encountered challenges in fully implementing the Rust-like
ownership and borrowing rules to ensure memory safety without a garbage collector.
The \borrowChecker{} managed ownership within functions but faced difficulties with
ownership transfer between functions. Consequently, this aspect of the project did
not fully meet the objectives set forth in sections~\ref{sec:Objectives} and
\ref{sec:Requirements}.



\newpage
