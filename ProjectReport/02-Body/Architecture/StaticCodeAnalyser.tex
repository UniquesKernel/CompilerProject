\subsection{Static Code Analyser}

Following the syntactical validation by the \lexer{} and \parser{}, which results in
the formation of an \ast{}, the next critical phase involves the \static{}. This
component is tasked with conducting a semantic analysis of the \ast{}, a process
essential for ensuring the semantic correctness of the program. \\

The \typeChecker{} is responsible for scrutinizing the use of types within the
program. It systematically verifies that each type is correctly declared and applied,
ensuring type consistency across the program. This step is vital in preventing
type-related errors and maintaining the integrity of data handling within the
program. \\

On the other hand, the \borrowChecker{} focuses on memory safety. It analyses the
borrowing and lifespan of variables, ensuring compliance with stringent
rules\footnote{These are the rules also used by Rust} for memory, and concurrency
safety\cite{RUST}. The \borrowChecker{} plays a crucial role in avoiding common pitfalls
related to managing memory, such as memory leaks, unauthorized access and dangling
pointers, thereby enhancing the program's reliability and safety. \\

In tandem, these two stages of semantic analysis form a comprehensive approach to
validating the program beyond its syntax, ensuring that the logic is sound and the
memory usage is safe. The following sections provide a detailed exploration of the
\typeChecker{} and \borrowChecker.

\subsubsection{Type Checker}

The \typeChecker{} 

\newpage

\subsection{Borrow Checker}
\label{sec:BorrowCheckerImpl}



