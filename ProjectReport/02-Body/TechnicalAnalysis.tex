\section{Technical Analysis}
\label{sec:TechAnalysis}

On the topic of technologies, as the \lang{} compiler borrows many features from the
Rust language, it's worth taking a look at the Rust compiler to see how Rust has
organized its compiler architecture. 

\begin{figure}[ht]
  \centering
  \begin{tikzpicture}

    % Define the words
    \def\words{{Input}, {Invocation}, {Lexing}, {Parsing}, {HIR lowering},
               {MIR lowering}, {Code Generation}, {Output}}
    \def\datas{{Source Code}, {Source Code}, {Token Stream}, {AST}, {HIR
    Representation}, {MIR Representation}, {Binary File}}
 
    \node[draw, rectangle, minimum height=1cm, minimum width=3cm] (node1) {Input};

    % Loop through each word
    \foreach \word [count=\i] in \words {
            % Node style for HIR and MIR lowering
      \pgfmathtruncatemacro{\prevIndex}{\i - 1}
      
      \ifnum\i=1\relax
      \else 
        \ifthenelse{\equal{\word}{HIR lowering} \OR \equal{\word}{MIR lowering}} {
          \node[draw, rectangle, minimum height=1cm, minimum width=3cm, below=of
            node\prevIndex, yshift=.5cm, anchor=north] (node\i) {\word};
        }{
          \node[draw, rectangle, minimum height=1cm, minimum width=3cm, below=of
            node\prevIndex, yshift=.5cm] (node\i) {\word};
        }
      \fi
    }

    \foreach \data [count=\i] in \datas {
      \ifnum\i=0\relax 
      \else
        \pgfmathtruncatemacro{\nextIndex}{\i + 1}
        \draw[->] (node\i) -- node[right]{\data} (node\nextIndex);
      \fi
    }
  \end{tikzpicture}
  \caption{Diagram architecture of the Rust Compiler}
\end{figure}

On a high level this is a very typical architecture for a compiler\cite{GEEK}. The
following sections take a look at each component in more details to determine if the
\lang{} compiler can benefit from any of these components and which tools, if any,
might be useful for implementing the specific components.

\subsection{Invocation}

This component deals primarily with the configuration of the Rust compiler. It's a
step where the Rust compiler is invoked on some source code, typically through the
\texttt{cargo} CLI tool\cite{RUST_COMPILER}. 

This phase does not directly influence the objectives of \lang{} and so it will not
be included due to time constraints. It is, however, worth noting the importance of
intuitive CLI tools when working with a compiler and it is a good candidate for
future work after the rest of the compiler phases is fully implemented.

\subsection{\lexer{}}

The \lexer{} is hugely important for any compiler as it is responsible for
reading the source code and producing the necessary tokens for the parser to later
construct the \ast{} representation of the program. The Rust compiler implements its
own lexical analyser from scratch, which after some experimentation is not too
difficult a task to accomplish as this can be done using a character stream along
with a list of regular expressions to search for each kind of token in the source
code. This would give great control of the output of the lexer ensuring a perfect
match to input required by the parser. 

The issue here is that the development of \lang{} is under some time constraints so
to avoid the lengthy process of writing and maintaining a \lexer{} from the buttom up
it's faster and safer to use a tool to generate the \lexer{}. One such tool is the
\lexerGen{} tool, which is a tool that is configured by providing the regular
expressions to search for along with the tokens defined in the parser, it is then
able to generate a token stream containing the specified tools. 

\subsubsection{Flex}

\lexerGen{} is a tool for generating \textit{lexers} or \textit{scanners} and is
often paired with \parserGen{} to produce \parser{}s and interpreters. It's an
open-source project and is highly valued for its speed and efficiency in text
processing. \\

The use of \lexerGen{} for the \lang{} project streamlines the process of lexical
analysis and makes it simple to create or alter the definition of lexemes. Its
compatibility with \parserGen{} ensures seamless integration between the \lexer{} and
the \parser. Just as with \parserGen, the choice of \lexerGen{} is reinforced by its
comprehensive documentation, active community support, and its proven track record in
various software projects. This makes it a reliable and robust choice for the lexical
analysis phase of the \lang{} compiler.

\subsection{Parser}
\label{sec:techParser}

The Rust language is build using a recursive decent parser, and just as with its
\lexer{} Rust implements this from scratch\cite{RUST_COMPILER_SRC}. This is one of
the simplest forms of parsers to write by hand, but it requires quite a lot of
maintainance if the nature of the syntax changes, or to add new features to a
language. Writing the parser from scratch does give full control over the input
and the output \ast{} generated by the \parser{}. For \lang{} this was explored
by spending two weeks trying to implement a more flexible CLR(1) parser, but this was
found to take away too much from the goal of the project. To keep the focus on the
objective of \lang{} it was instead decided that it would be better to use a tool to
generate a \parser{}.

\subsubsection{\parserGen{}}

\parserGen{} is a prominent parser generator, part of the GNU project, capable of
producing parsers for a variety of languages by supporting multiple parsing
algorithms\cite{BISON}. 

It works by defining tokens the \lexer{} should generate along with production rules
following symantics similarly to the Backus–Naur form (BNF). It then generates a
parser which can parse the specified syntax.

For the \lang{} compiler, \parserGen{} was employed because of its flexibility,
efficiency, and user-friendliness. It facilitates rapid iterations and adjustments to
the grammar. Moreover, its robust and modular parser implementation can be easily
extended. Notably, \parserGen{} integrates natively with \lexerGen{}, ensuring a
cohesive interaction between the \lexer{} and \parser{}.

When configured appropriately, \parserGen{} can generate a \textit{\parserType{}}
parser using LALR(1), IELR(1), CLR(1), or GLR(1) parsing tables these are largly the
same types of parsers but some are more general than others. Its comprehensive
documentation and active community support further cemented its selection as the
parser generator of choice for the \lang{} compiler.

\subsection{HIR and MIR lowering}
\label{sec:hirmir}

These steps effectively correspond to a \static{}. Rust, however, takes the approach
to convert it's \ast{} into different intermediate representations namely a
high-level and medium-level intermediate representation, which it claims makes type
checking and borrow checking simpler to perform in contrast to working directly on
the \ast. 

\lang{} doesn't implement these types of conversions and instead works directly with
the \ast{} when doing type checking and borrow checking and collects these steps on a
single architectural step, the \static. This is done to minimize complexity due to
the limited objectives that \lang{} is attempting to achieve (see
section~\ref{sec:Objectives}).

\subsection{Code Generation}
\label{sec:codeGenTech}

Rust makes use of LLVM as its compiler backend to transform its intermediate
representations into excutable files. LLVM allows us to transform out \ast{} into
LLVM IR code which it can then optimize and use to produce object files, executables
etc. The alternative here is to write a compiler backend from scratch which would be
incredibly difficult and prevent any form of cross-compilation unless the custom
backend is written to support multiple different architecture. 

\lang{} will also be using LLVM here for the convenience it provides in transforming
\ast{} structures into low-level machine instructions. Unlike Rust, which uses LLVM
to create it's own runtime, \lang{} will instead use LLVM to build object files which
can utilize the \gcc{} compiler tool to tab into various standard librarys and
runtime elements. Further, more by using the \gcc{} this allows \lang{} be linked and
called inside \texttt{C}/\texttt{C++} code. It also allows \lang{} access to various
optimizations provided by the \gcc.


\subsection{Conclusion}
\label{sec:techCon}

??


\newpage
