\section{Process}

Our process has been a loosely coordinated heuristic process. Using the fact that the
group had only two members, and that the members were familiar with working together
allowed us to dispense with a formal group structure, and to handle all coordination
in weekly meetings and through conversation. The group used discord to facilitate
communication, and github for version control of the source code.\\

An initial roadmap was made before beginning the development where the overall
stages in the process was marked out and give an expected duration(see figure
\ref{fig:iterate}). The roadmap was made as a first estimate of the steps required to
answer the problem statement. 

\begin{figure}[h]
\centering
\includegraphics[width=\textwidth]{02-Body/Images/Roadmap.png}
\caption{Initial roadmap for the project}
\label{fig:iterate}
\end{figure}

The iterative process consisted of weekly meetings where the current state of
development was reviewed, and future work discussed and planned. To determine which
tasks should be focused on in the next iteration, we considered what would be the
smallest step towards answering the problem statement, while leaving us with a
'finished' product at the end of each iteration. This could be tasks such as:

\begin{itemize}
  \item Implement the Type Checker for the current state of \lang{}.
  \item Implement Function calls.
\end{itemize} 

With this approach we always had a working product, and if something did not work out
as expected, only one or at most two weeks of work would be lost. The roadmap illustrates
that the majority of the development was done as two parallel tasks. This was done to
maximise the work done, minimize duplicate work, and avoid merge conflicts in git.

\newpage
