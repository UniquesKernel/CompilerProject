\section{Language Syntax}
\label{sec:syntax}
NOTE: things as expressions?
\subsection{Variables}

let mut Identifier: type = value\\

Identifier ::= StartChar SubsequentChars*\\
StartChar      ::= [a-zA-Z\_]\\
SubsequentChars::= [a-zA-Z0-9\_]\\

type: [int, float, char, bool]

\subsection{Expressions}


\subsection{Conditionals}

IfExpression ::= "if" expression Block ?ElseExpression?

ElseExpression ::= "else" (Block | IfExpression)

let a:int = if(bool) return 5

fn fun():void{

 let a:int = 1
 
 if(!test()) {return }
 
 do something else
 

}





let a = fun

else if block

else block




\subsection{Functions}
FunctionExpr      ::= "fn" FunctionName ParamList ":" type Block\\
FunctionName	  ::= Identifier
ParamList         ::= "(" Param ( "," Param )* ")"\\
Param             ::= Identifier ":" Type\\
Block             ::= "{" Statement* "}"\\
Statement         ::= /* definition of what constitutes a statement in your language /\\
Type              ::= / definition of what constitutes a type in your language /\\
Identifier ::= StartChar SubsequentChars*\\
StartChar      ::= [a-zA-Z\_]\\
SubsequentChars::= [a-zA-Z0-9\_]\\

fn func(param:bool):int {code}\\


fn test(param\_ fun:(name:int, var:int)-\> int):(int,int)-\> int {\\
param\_ fun(1,1)\\
return param\_ fun\\
}\\

let a:(int,int)-\> int = test(func)\\

\subsection{Types}
int: 32-bit integer\\
float: 64-bit floating point\\
char: 8-bit integer\\
bool: 8-bit\\
void
