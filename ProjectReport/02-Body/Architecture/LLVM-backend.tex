\subsection{\texttt{C}ode Generator and the \gcc{}}
\label{sec:LLVM}

The \codeGen{} for the \lang{} compiler is responsible for traversing the
\ast{} and translating each node into its corresponding low-level
instructions. This process involves meticulously converting the structured,
high-level representations of the \ast{} into a series of machine-level instructions.
Following this, the \codeGen{} outputs this low-level code into an object file.
This file, while containing machine-level code, is not yet in a form that can be
directly run on a computer. \\

At this juncture, the \gcc{} plays a vital role. It
takes the object file produced by our \texttt{C}ode Generator and performs the necessary
linking with other object files. This step is essential for resolving external
references and integrating various code modules into a cohesive whole. After linking,
the \gcc{} embarks on translating the aggregated object code into an executable file. \\

This use of the \gcc{} offers three distinct advantages: 

\begin{itemize}
  \item[\textbf{Optimization:}] The \gcc{} is capable of very sophisitcated
    optimizations on our low-level code.
  \item[\textbf{Target Specific \texttt{C}ompilation:}] The \gcc{} is capable of generating code
    for a wide target of hardware architectures allowing \lang{} code to run on must
    platoforms and architectures.
  \item[\textbf{Integration with \texttt{C}/\texttt{C++}:}] \texttt{C}/\texttt{C++} also make use of the \gcc{} and
    object files. By forwarding function declaration in \texttt{C}/\texttt{C++}, functions written in
    \lang{} can be forward declared and called inside \texttt{C}/\texttt{C++} code
    bases. This, in theory, allows
    users to take advantage of \lang{}'s memory and concurrency
    safety\footnote{This has been observed and excuted for simple programs but
    not tested in full}.
\end{itemize}

