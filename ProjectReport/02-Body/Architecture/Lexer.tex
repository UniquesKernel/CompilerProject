\subsection{Lexer}

The \lexer{} processes the source code as a stream of characters. Once it identifies
a valid lexeme, it emits the corresponding token. Specifically, the \lexer{} follows
the maximal munch rule, recognizing the longest verified lexeme, ensuring that tokens
represent the most appropriate grouping of characters. In fine, the \lexer{}
transforms a stream of characters into a stream of tokens.

\section{Appendix A}
\label{sec:appA}

\begin{figure}[ht]
\centering
  \midsepremove{}
  \begin{tabular}{|l|l|l|}
    \toprule
    Symbol & Token & Regular Expression \\
    \midrule
    "fn" & FUNCTION & \regex{"fn"} \\
    "true", & T\_TRUE & \regex{"true"} \\
    "false" & T\_FALSE & \regex{"false"} \\
    "return" & RETURN & \regex{"return"} \\
    "if"  & IF\_TOKEN & \regex{"if"} \\
    "else" & ELSE\_TOKEN & \regex{"else"} \\
    "let" & KW\_VAR & \regex{"let"} \\
    "mut" & KW\_MUT & \regex{"mut"} \\
    \hline
    "\&int" & TYPE\_REF & \regex{"&int"} \\
    "\&bool" & TYPE\_REF & \regex{"&bool"} \\
    "\&float" & TYPE\_REF & \regex{"&float"} \\
    "\&char" & TYPE\_REF & \regex{"&char"} \\
    "int" & TYPE & \regex{"int"} \\
    "bool" & TYPE & \regex{"bool"} \\
    "float" & TYPE & \regex{"float"} \\
    "char" & TYPE & \regex{"char"} \\
    "\&" & KW\_REF & \regex{"&"} \\
    "\&mut" & KW\_MUT\_REF & \regex{"&mut"} \\
    \hline
    "==", "!=" & EQ, NEQ & \regex{"=="}, \regex{"!="} \\ 
    "\l", "\g" & LT, GT & \regex{"<"}, \regex{">"} \\
    "*", "/" & MUL, DIV & \regex{"*"}, \regex{"/"} \\
    "+", "-" & PLUS, MINUS & \regex{"+"}, \regex{"-"} \\
    "\%" & MOD & \regex{"\%"} \\
    \hline
    Identifiers & IDENTIFIER & \regex{[a-zA-Z][a-zA-Z0-9]*} \\
    Character literals & TOKEN\_CHAR & \regex{[a-zA-Z]} \\
    Floating point numbers & TOKEN\_FLOAT & \regex{[0-9]+\.[0-9]+} \\
    Integer literals & TOKEN\_INT & \regex{[0-9]+} \\
    \hline
    "\{" "\}" & LBRACE, RBRACE & \regex{"{" "}"} \\
    ";" & END\_OF\_LINE & \regex{[;]} \\
    ":" & COLON & \regex{":"} \\
    "," & COMMA & \regex{[,]} \\
    "(" ")" "=" & LPAREN, RPAREN, ASSIGN & \regex{"("} \regex{")"} \regex{"="} \\
    tabs, spaces, and newlines & (Ignored) & \regex{[ \\t\\n\\r]} \\
    Anything else & Throws error: "Unknown Token" & \regex{[.]} \\
    \bottomrule
  \end{tabular}
  \caption{The \lexer{} recognizes lexemes by using regular
  expressions. The table shows which lexemes are mapped to which tokens and which
regular expression is used to identify them.}
  \label{fig:SymbolMap}
\end{figure}



\vspace*{-.5cm}
\subsubsection{Flex}

\lexerGen{} is a tool for generating \textit{lexers} or \textit{scanners} and is
often paired with \parserGen{} to produce compilers and interpreters. It's an
open-source project and is highly valued for its speed and efficiency in text
processing. \\

The use of \lexerGen{} for the \lang{} project streamlines the process of lexical
analysis and makes it simple to create or alter the definition of lexemes. Its
compatibility with \parserGen{} ensures seamless integration between the \lexer{} and
\parser{}. Just as with Bison, the choice of \lexerGen{} is reinforced by its
comprehensive documentation, active community support, and its proven track record in
various software projects. This makes it a reliable and robust choice for the lexical
analysis phase of the \lang{} compiler.

