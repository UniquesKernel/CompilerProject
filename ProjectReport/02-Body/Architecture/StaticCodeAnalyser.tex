\subsection{Static Code Analyser}

Following the syntactical validation by the \lexer{} and \parser{}, which results in
the formation of an \ast{}, the next critical phase involves the \static{}. This
component is tasked with conducting a semantic analysis of the \ast{}, a process
essential for ensuring the semantic correctness of the program. The \static{}
contains two subcomponents, the \typeChecker{} and the \borrowChecker{} which each is
responsible for one of two phases in the \static.\\

The \typeChecker{} is responsible for scrutinizing the use of types within the
program. It systematically verifies that each type is correctly declared and applied,
ensuring type consistency across the program. This step is vital in preventing
type-related errors and maintaining the integrity of data handling within the
program. \\

On the other hand, the \borrowChecker{} focuses on memory safety. It analyses the
borrowing and lifespan of variables, ensuring compliance with stringent
rules\footnote{These are the rules also used by Rust} for memory, and concurrency
safety\cite{RUST}. The \borrowChecker{} plays a crucial role in avoiding common
pitfalls related to managing memory, such as memory leaks, unauthorized access and
dangling pointers, thereby enhancing the program's reliability and safety. It also
ensures safety when dealing with concurrent programs as specific rules are design to
prevent race conditions.\\

In tandem, these two stages of semantic analysis form a comprehensive approach to
validating the program beyond its syntax, ensuring that the logic is sound and the
safe allocation and deallocation of memory. The following sections provide a detailed
exploration of the \typeChecker{} and \borrowChecker.

\subsection{Type Checker}
The type checker implements a visitor, as described in section \ref{sec:typeCheckerDesign}. To the types available are \textit{int}, \textit{float}, \textit{char} and \textit{bool}. To keep track of the type of each expression, the base expression (that all expressions inherit from) was given an attribute "type", with public get and set functions. The types where denoted with string types. It was considered using an Enumerator, but in order to get usable error messages from the \lang{} compiler, it was necessary to have the types as string anyway. Having the types as strings also allows for easy expansion of the number of permitted types, as only terminal expressions determines type.\\
Variables are handled with a symbol table that maps the variable identifier to its type. As it is possible to have several variables with the same name if they are in different scopes, the symbol table is actually implemented as a stack of maps, each corresponding to  a single scope. This is explained in more detail in section \ref{sec:CodeGenImplement}, as it is the same technique used in Code Generation.\\
\subsubsection{Type in AST}
The code generator and borrow checker need to know the types of certain expressions, which poses a small problem. Most expressions do not know what their type is. I.e. a binary expression knows only which two expressions to work on, but not what type they are. Only terminal expressions know their own type. In order for the borrow checker and code generator to work, each expression therefore must be given a type. This task falls to the typechecker. As it checks that the type safty is kept, it calculates the type of every expression, and, once certain that the expression does not violate type safety, sets the type field in the expression to the calculated type.\\
This introduces a bit of coulpling, as the later modules depend on the typechecker setting the types in the AST, but the alternative is to calculate the types when creating the tree, and again in the typechecker, it was decided that a slight coulping was preferable to this double work. It is of coures still possible to use the modularity that the visitor pattern gives, as one can easily still swap the typechecker with another, as long as the new typechecker also writes the types to the AST.
 
\subsection{Borrow Checker}
\label{sec:BorrowCheckerImpl}

The \borrowChecker{} implements the visitor pattern, as described in
section~\ref{sec:VisitorDesign}, and uses two tables; a
\texttt{symbolTable} and a \texttt{referenceTable} to keep track of variables as well
as keeping track of references, as described in
Section~\ref{sec:BorrowCheckerDesign}. The \borrowChecker{} then uses these tables to enforce the
same ownership, borrowing and reference rules as the \texttt{Rust} language as described
in section~\ref{sec:LanguageDesign} and section~\ref{sec:BorrowCheckerDesign}. \\

\todo{re-write snippet to add more to the implementation instead of discussions}

Consider the code snippet as provided in listing~\ref{lst:referenceExample}. When the
\borrowChecker{} needs to evaluate if a reference assignment is legal, it retrieves
the a list of existing references to the variable that is being referenced from the
\texttt{referenceMap} and then does some checks to ensure that the \ast{} is in
compliance with the rules laid out in section~\ref{par:Ownership}.

\begin{lstlisting}[
  language=c++,
  numbers=left,
  frame=single,
  caption={Code showing a snippet of the referenceAssignment function in the \lang{}
  \borrowChecker{} implementation},
  label={lst:referenceExample},
  showstringspaces=false,
  ]
  ... // validation of the existence of the referenced variable 
  ... // cut out for brevity. 

   std::vector<std::pair<std::string, bool>> &vec =
      referenceMap[variable->getReferenceIdentifier()]; 

  int mutableCount = 0;
  bool hasImmutable = false;
  for (const auto &pair : vec) {
    if (pair.second) {
      mutableCount++;
    } else {
      hasImmutable = true;
    }
  }

  if (mutableCount > 1) {
    throw std::invalid_argument(
        "Invalid argument: more than one mutable reference");
  } else if (mutableCount == 1 && hasImmutable) {
    throw std::invalid_argument(
        "Invalid argument: mixed mutable and immutable references");
  }
  ...
\end{lstlisting}

This implementation is only responsible for tracking references inside a single
function scope, the implementation gets more difficult when tracking variables
between function scope and the current implementation of the \borrowChecker{} does
not yet accomplish this leading to correct programs throwing compile time errors when
they should. \\

The issue here seems to stem from an issue of correctly tracking and removing
variables from the \texttt{symbolTable} and the \texttt{referenceTable} when the
scopes are not nested inside each other.

