\section{Future Work}
\label{sec:FutureWork}

Considering the results as laid out in Section~\ref{sec:Discussion}; what work is left
unfinished and what else can then be added to the scope of the project in the future?
For a good overview of the possible future work let's split this into three sections;
unfinished work, feature work and tooling.

\subsubsection{Unfinished Work}
\label{sec:unfinished}

When looking strictly at the objectives of the project, the \borrowChecker{} is the
single largest missing piece of work. To fully complete the project's objective the
current issues with the \borrowChecker{} needs to be fixed. 

Further, while not strictly related to the objectives of the project, having the
ability to read inputs at runtime, is a major missing feature when considering the
current state of the project.

\subsubsection{Feature Work}
\label{sec:feature}

\lang{} is a very minimalistic language and so it lacks many features that is
currently taken for granted in many languages, among these features are loops,
arrays, and ways to combine associated data types, like algebraic types like in
\texttt{Haskell} or structs like in \texttt{C}.\\

\lang{} currently lack any support for heap allocated memory which is a very powerful
feature of most modern programming languages. The inclusion of more data types like
fixed size array is also a natural next step for \lang's features. With heap
allocations come a whole new set of rules for the \borrowChecker{} which it would
need to be extended to handle. \\

To help better organize the code base adding a way of supporting multiple files and
importing functions and other future features into the files where they are relevant
would be a huge step when it comes to new features. \\

The creation of a standard library, would help \lang{} significantly as it would help
trivialize many mundane operations like working with arrays, lists and maps. \\

Error handling is a concept that \lang{} does not support at all, currently.
Adding some kind of stucture to handle errors, in the form of monads or exception
handling would be a crucial feature, and with \lang's focus on explicitness and
safety it is also a very natural concept to add to the language at large.\\

One feature related to the use of LLVM which would act as a natural extension to its
current use would be to allow \lang{} to use its own runtime setup rather than the
setup provided by the \gcc{} which would allow PIE (see section~\ref{sec:Output}) to be
re-enabled which would be very desirable when the language is extended to allow user
input as it would help prevent certain vulnerabilities.

\subsubsection{Tooling} \label{sec:tooling}

Any language and compiler needs great tooling to support the ease of using the
language. Language like \texttt{Rust} have the Cargo tool, \texttt{C/C++} have tools like make
and cmake, \texttt{Haskell} have the stack and cabal tools. When considering any
future work, tools like these are therefore natural to consider for \lang{} to help
manage dependencies, install dependencies and configure the compiler, with debugging,
release, and optimization flags. 

