\section{Architecture}
\label{sec:architecture}

On the topic of architecture, the \lang{} compiler is comprised of five primary
components: the \lexer, the \parser, the \static, the \codeGen{} and the \gcc. These
components form a pipeline, where the output of one component is the input of the
next; in tandem, they constitute the complete compilation process. \\

\lang{} is a strong statically typed language with a focus on memory safety. It's
explicit type system is designed to minimise confusion and obfuscation as each
variable must be explicitly declared with a type. It's memory safety is heavily
inspired by Rust, and is achieved through the use of a \borrowChecker{} which enforces
a subset of Rust's borrow checking rules\footnote{The subset of rules enforced
exclude memory on the heap as the language is yet to support heap allocations}. \\

\begin{figure}[h]
  \begin{center}
    \begin{tikzpicture}

      \node[draw, rectangle, minimum width=4cm] (source) {Source Code};
      \node[draw, rectangle, minimum width=4cm, below=0.7cm of source] (lexer) {\lexer};
      \node[draw, rectangle, minimum width=4cm, below=0.7cm of lexer] (parser) {\parser};
      \node[draw, rectangle, minimum width=4cm, below=0.7cm of parser] (semantic) {\semantic};
      \node[draw, rectangle, minimum width=4cm, below=0.7cm of semantic] (codeGen) {\codeGen};
      \node[draw, rectangle, minimum width=4cm, below=0.7cm of codeGen] (output) {Executable};

      \draw[->] (source) -- (lexer) node[midway, right] {Source File};
      \draw[->] (lexer) -- (parser) node[midway, right] {Token Stream};
      \draw[->] (parser) -- (semantic) node[midway, right] {AST};
      \draw[->] (semantic) -- (codeGen) node[midway, right] {AST};
      \draw[->] (codeGen) -- (output) node[midway, right] {IR};
    \end{tikzpicture}
  \end{center}
  \caption{The \lang{} compiler pipeline, transforming source code into an executable.}
  \label{fig:CompilerProcess}
\end{figure}



To facilitate the \static{} and the \codeGen{} in a way that is both efficient and
modular, the \lang{} compiler is written using a visitor pattern. This pattern
efficiently enables the modification of the \ast{} without requiring extensive
modifications to the implementation of the \ast. Thus if changes are made to the
\lang{} language, such as grammar or syntax modification, it seldom requires the
\ast{} implementation to be modified. \\

Once the code has passed the \static{}; the \codeGen{} will translate the \ast{} into
an object file and the \gcc{} will link the object file and produce a machine excutable file.

\newpage

\subsection{Lexer}

The \lexer{} processes the source code as a stream of characters. Once it identifies
a valid lexeme, it emits the corresponding token. Specifically, the \lexer{} follows
the maximal munch rule, recognizing the longest verified lexeme, ensuring that tokens
represent the most appropriate grouping of characters. In fine, the \lexer{}
transforms a stream of characters into a stream of tokens.

\section{Appendix A}
\label{sec:appA}

\begin{figure}[ht]
\centering
  \midsepremove{}
  \begin{tabular}{|l|l|l|}
    \toprule
    Symbol & Token & Regular Expression \\
    \midrule
    "fn" & FUNCTION & \regex{"fn"} \\
    "true", & T\_TRUE & \regex{"true"} \\
    "false" & T\_FALSE & \regex{"false"} \\
    "return" & RETURN & \regex{"return"} \\
    "if"  & IF\_TOKEN & \regex{"if"} \\
    "else" & ELSE\_TOKEN & \regex{"else"} \\
    "let" & KW\_VAR & \regex{"let"} \\
    "mut" & KW\_MUT & \regex{"mut"} \\
    \hline
    "\&int" & TYPE\_REF & \regex{"&int"} \\
    "\&bool" & TYPE\_REF & \regex{"&bool"} \\
    "\&float" & TYPE\_REF & \regex{"&float"} \\
    "\&char" & TYPE\_REF & \regex{"&char"} \\
    "int" & TYPE & \regex{"int"} \\
    "bool" & TYPE & \regex{"bool"} \\
    "float" & TYPE & \regex{"float"} \\
    "char" & TYPE & \regex{"char"} \\
    "\&" & KW\_REF & \regex{"&"} \\
    "\&mut" & KW\_MUT\_REF & \regex{"&mut"} \\
    \hline
    "==", "!=" & EQ, NEQ & \regex{"=="}, \regex{"!="} \\ 
    "\l", "\g" & LT, GT & \regex{"<"}, \regex{">"} \\
    "*", "/" & MUL, DIV & \regex{"*"}, \regex{"/"} \\
    "+", "-" & PLUS, MINUS & \regex{"+"}, \regex{"-"} \\
    "\%" & MOD & \regex{"\%"} \\
    \hline
    Identifiers & IDENTIFIER & \regex{[a-zA-Z][a-zA-Z0-9]*} \\
    Character literals & TOKEN\_CHAR & \regex{[a-zA-Z]} \\
    Floating point numbers & TOKEN\_FLOAT & \regex{[0-9]+\.[0-9]+} \\
    Integer literals & TOKEN\_INT & \regex{[0-9]+} \\
    \hline
    "\{" "\}" & LBRACE, RBRACE & \regex{"{" "}"} \\
    ";" & END\_OF\_LINE & \regex{[;]} \\
    ":" & COLON & \regex{":"} \\
    "," & COMMA & \regex{[,]} \\
    "(" ")" "=" & LPAREN, RPAREN, ASSIGN & \regex{"("} \regex{")"} \regex{"="} \\
    tabs, spaces, and newlines & (Ignored) & \regex{[ \\t\\n\\r]} \\
    Anything else & Throws error: "Unknown Token" & \regex{[.]} \\
    \bottomrule
  \end{tabular}
  \caption{The \lexer{} recognizes lexemes by using regular
  expressions. The table shows which lexemes are mapped to which tokens and which
regular expression is used to identify them.}
  \label{fig:SymbolMap}
\end{figure}



\vspace*{-.5cm}
\subsubsection{Flex}

\lexerGen{} is a tool for generating \textit{lexers} or \textit{scanners} and is
often paired with \parserGen{} to produce compilers and interpreters. It's an
open-source project and is highly valued for its speed and efficiency in text
processing. \\

The use of \lexerGen{} for the \lang{} project streamlines the process of lexical
analysis and makes it simple to create or alter the definition of lexemes. Its
compatibility with \parserGen{} ensures seamless integration between the \lexer{} and
\parser{}. Just as with Bison, the choice of \lexerGen{} is reinforced by its
comprehensive documentation, active community support, and its proven track record in
various software projects. This makes it a reliable and robust choice for the lexical
analysis phase of the \lang{} compiler.


\subsection{Parser}

The \lang{} \parser{} utilizes a \textit{bottom-up} parsing methodology.
Specifically, it is a \parserType{}\footnote{Look-Ahead LR Parser with 1 symbol of
lookahead} parser that is capable of parsing all deterministic context-free
grammars (CFGs).

Given a grammar, the \parser{} synthesizes a \textit{parse table} that is used to
determine what action to take based on the current \textit{state}\footnote{The tokens
read so far} and the next token in the token stream. 

The \parser{} can take one of four actions: \textit{shift}, \textit{reduce},
\textit{accept}, or \textit{error}. 

\begin{itemize} 
  
  \item \textbf{Shift} - When the \parser{} reads a token that is not the last token
    in a production rule, it shifts. This means it places the current token on a
    stack, reads the next symbol, and transitions to the state specified by the parse
    table.

  \item \textbf{Reduce} - When the \parser{} reads a token that is the last token in
    a production rule, it reduces. This involves popping the symbols on the stack
    that correspond to the production rule, pushing the non-terminal symbol specified
    by the production rule onto the stack, and transitioning to the state specified
    by the parse table.

  \item \textbf{Accept} - The \parser{} enters this state when it has successfully
    parsed the entire token stream.

  \item \textbf{Error} - This state is entered when the \parser{} encounters an
    unexpected token that is not valid in any state reachable from the current state.

\end{itemize}

This means the \parser{} effectively becomes an automaton, cycling through states
shifting and reducing until it either accepts or errors. In this process it builds an
\textit{abstract syntax tree} (AST) that represents the structure of the program.

\subsubsection{Bison}

\parserGen{} is a prominent parser generator, part of the GNU project, capable of
producing parsers for a variety of languages by supporting multiple parsing
algorithms\cite{BISON}. 

For the \lang{} compiler, \parserGen{} was employed because of its flexibility,
efficiency, and user-friendliness. It facilitates rapid iterations and adjustments to
the grammar. Moreover, its robust and modular parser implementation can be easily
extended. Notably, \parserGen{} integrates natively with \lexerGen{}, ensuring a
cohesive interaction between the \lexer{} and \parser{}.

When configured appropriately, \parserGen{} can generate a \textit{\parserType{}}
parser using LALR(1), IELR(1), or CLR(1) parsing tables. Its comprehensive
documentation and active community support further cemented its selection as the
parser generator of choice for the \lang{} compiler.


\subsection{Static Code Analyser}

Following the syntactical validation by the \lexer{} and \parser{}, which results in
the formation of an \ast{}, the next critical phase involves the \static{}. This
component is tasked with conducting a semantic analysis of the \ast{}, a process
essential for ensuring the semantic correctness of the program. The \static{}
contains two subcomponents, the \typeChecker{} and the \borrowChecker{} which each is
responsible for one of two phases in the \static.\\

The \typeChecker{} is responsible for scrutinizing the use of types within the
program. It systematically verifies that each type is correctly declared and applied,
ensuring type consistency across the program. This step is vital in preventing
type-related errors and maintaining the integrity of data handling within the
program. \\

On the other hand, the \borrowChecker{} focuses on memory safety. It analyses the
borrowing and lifespan of variables, ensuring compliance with stringent
rules\footnote{These are the rules also used by Rust} for memory, and concurrency
safety\cite{RUST}. The \borrowChecker{} plays a crucial role in avoiding common
pitfalls related to managing memory, such as memory leaks, unauthorized access and
dangling pointers, thereby enhancing the program's reliability and safety. It also
ensures safety when dealing with concurrent programs as specific rules are design to
prevent race conditions.\\

In tandem, these two stages of semantic analysis form a comprehensive approach to
validating the program beyond its syntax, ensuring that the logic is sound and the
safe allocation and deallocation of memory. The following sections provide a detailed
exploration of the \typeChecker{} and \borrowChecker.

\subsubsection{Type Checker}

The \typeChecker{} 

\newpage

\subsection{Borrow Checker}
\label{sec:BorrowCheckerImpl}




\subsection{Code Generator and the \gcc{}}
\label{sec:LLVM}

The \codeGen{} for the \lang{} compiler is responsible for traversing the
\ast{} and translating each node into its corresponding low-level
instructions. This process involves meticulously converting the structured,
high-level representations of the \ast{} into a series of machine-level instructions.
Following this, the \codeGen{} outputs low-level code into an object file.
This file, while containing machine-level code, is not yet in a form that can be
directly run on a computer. \\

At this juncture, the \gcc{} plays a vital role. It
takes the object file produced by our \codeGen{} and performs the necessary
linking with other object files. This step is essential for resolving external
references and integrating various code modules into a cohesive whole. After linking,
the \gcc{} embarks on translating the aggregated object code into an executable file. \\

\newpage

This use of the \gcc{} offers three distinct advantages: 

\begin{itemize}
  \item[\textbf{Optimization:}] The \gcc{} is capable of very sophisitcated
    optimizations on our low-level code.
  \item[\textbf{Target Specific Compilation:}] The \gcc{} is capable of generating code
    for a wide target of hardware architectures allowing \lang{} code to run on most
    platforms and architectures.
  \item[\textbf{Integration with \texttt{C}/\texttt{C++}:}] \texttt{C}/\texttt{C++} also make use of the \gcc{} and
    object files. By forwarding function declaration in \texttt{C}/\texttt{C++}, functions written in
    \lang{} can be forward declared and called inside \texttt{C}/\texttt{C++} code
    bases. This, in theory, allows
    users to take advantage of \lang{}'s memory and concurrency
    safety\footnote{This has been observed and excuted for simple programs but
    not tested in full} while writting \texttt{C}/\texttt{C++} programs.
\end{itemize}



\newpage
