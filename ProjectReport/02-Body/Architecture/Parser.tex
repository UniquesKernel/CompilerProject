\subsection{Parser}

The \lang{} \parser{} utilizes a \textit{bottom-up} parsing methodology.
Specifically, it is a \parserType{}\footnote{Look-Ahead LR Parser with 1 symbol of
lookahead} parser that is capable of parsing all deterministic context-free
grammars (CFGs).

Given a grammar, the \parser{} synthesizes a \textit{parse table} that is used to
determine what action to take based on the current \textit{state}\footnote{The tokens
read so far} and the next token in the token stream. 

The \parser{} can take one of four actions: \textit{shift}, \textit{reduce},
\textit{accept}, or \textit{error}. 

\begin{itemize} 
  
  \item \textbf{Shift} - When the \parser{} reads a token that is not the last token
    in a production rule, it shifts. This means it places the current token on a
    stack, reads the next symbol, and transitions to the state specified by the parse
    table.

  \item \textbf{Reduce} - When the \parser{} reads a token that is the last token in
    a production rule, it reduces. This involves popping the symbols on the stack
    that correspond to the production rule, pushing the non-terminal symbol specified
    by the production rule onto the stack, and transitioning to the state specified
    by the parse table.

  \item \textbf{Accept} - The \parser{} enters this state when it has successfully
    parsed the entire token stream.

  \item \textbf{Error} - This state is entered when the \parser{} encounters an
    unexpected token that is not valid in any state reachable from the current state.

\end{itemize}

This means the \parser{} effectively becomes an automaton, cycling through states
shifting and reducing until it either accepts or errors. In this process it builds an
\textit{abstract syntax tree} (AST) that represents the structure of the program.

\subsubsection{Bison}

\parserGen{} is a prominent parser generator, part of the GNU project, capable of
producing parsers for a variety of languages by supporting multiple parsing
algorithms\cite{BISON}. 

For the \lang{} compiler, \parserGen{} was employed because of its flexibility,
efficiency, and user-friendliness. It facilitates rapid iterations and adjustments to
the grammar. Moreover, its robust and modular parser implementation can be easily
extended. Notably, \parserGen{} integrates natively with \lexerGen{}, ensuring a
cohesive interaction between the \lexer{} and \parser{}.

When configured appropriately, \parserGen{} can generate a \textit{\parserType{}}
parser using LALR(1), IELR(1), or CLR(1) parsing tables. Its comprehensive
documentation and active community support further cemented its selection as the
parser generator of choice for the \lang{} compiler.

