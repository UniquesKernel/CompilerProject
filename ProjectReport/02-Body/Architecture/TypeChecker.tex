\subsubsection{Type Checker}

The \typeChecker{} is responsible for the first phase of the \static. This involves
checking the \ast{} for type consistency. Using a visitor pattern and a bottom-up
approach, it verifies the type of each node in the \ast{} and ensures that the types
are consistent across the program. This is achieved by traversing the AST and
verifying that the expected type of each node is consistent with the type of its
children. \\

Once the \typeChecker{} verifies the consistency of a node, it annotates the node
with the expected type. This is done to facilitate the \codeGen{} component, which
uses the annotated type information to generate the correct intermediate
representation of the source code.

The types of the nodes will be verified using the following rules:
\begin{enumerate}

\item All nodes has the type of their child with following exceptions/clarifications:
\begin{enumerate}
\item Terminal expressions and variables determine their own type
\item Binary Expression Comparisons ($<$, $>$, $==$ and $!=$) has type \textit{bool}
\item Binary Expression involving references are automatically converted to the
  primitive type and not to a reference type\footnote{this is currently implemented
  they other way around}.
\item Blocks get the type of their return statements. 
\item If statements and functions has the type of the block(s)
\end{enumerate}
\item Binary expressions: Both children must have the same type
\item Binary expressions: References are threated as same type as the
  variable they reference.
\item All return statements in a block must have the same type
\item If/else statement blocks must be the same type
\item Variables must be assigned a value of its declared type
\item Function block type must match declared function return type
\end{enumerate}
