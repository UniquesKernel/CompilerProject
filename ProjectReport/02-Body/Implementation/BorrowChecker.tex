\subsection{Borrow Checker}
\label{sec:BorrowCheckerImpl}

The \borrowChecker{} implements the visitor pattern, as described in
section~\ref{sec:VisitorDesign}, and uses two tables a
\texttt{symbolTable} and a \texttt{referenceTable} to keep track of variables as well
as keeping track of references, as described in
Section~\ref{sec:BorrowCheckerDesign}. the \borrowChecker{} then uses these tables to enforce the
same ownership, borrowing and reference rules as the rust language as described
in Section~\ref{sec:LanguageDesign} and Section~\ref{sec:BorrowCheckerDesign}. \\

The implementation of the \borrowChecker{} is quite difficult and its current state
does not yet support all features in \lang{}. \\

The \borrowChecker{} does a good job tracking ownership and references within
functions, even across scopes. It it able to accurately track ownership and
references across internal scopes\footnote{None function call related changes in
scope, e.g. nested if-else statements and code blocks.} but once a variable is parsed
to a function as an argument the current implementation of the \borrowChecker{} has
difficulties keeping track of ownerships, in cases were the ownership of the
variables need to change, or references in cases were references are parsed as
arguments to a functions.

In fine, the \borrowChecker{} is not currently working as intended and due to this
issue the project does not deliver on its objective to provide memory safety without
the use of a garbage collector. Further more, due to the issues with the
\borrowChecker{} the compiler chain (see section~\ref{sec:architecture}) currently
causes some valid programs to not compile as intended.
