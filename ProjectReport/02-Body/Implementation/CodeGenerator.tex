\subsection{Code Generator}
\label{sec:CodeGenImplement}

\subsubsection{LLVM}


\subsubsection{Variables}


\subsubsection{Output}
Implementing an output posed a new type of challenge, as writing an output is not the responsibillity of the compiler, but rather the opperating system. As such the challenge went from finding the correct LLVM method to finding out how to use external functions in our program. Because an output is not strictly necessary to answer our problem statement, but more a quality of lif, it did become a quick and unappealing implementation, that needs refinement.\\
The first way to output elements from our  program involved another programming language (C). The fact that we compile to a .o file allowed us to write our functions, and then link the .o file to a C program that would call the \lang{} functions and print their output with the C implementation of printf(). A workable, but unsatisfying solution, that allowed us to test the compiler. The next step was to bring printf into \lang. Because our focus is on memory and type safety, the printfuntion implementation became a hardcoded reference to printf() with the specific formating required to print our types.\\
This works only because we already compile with libc linked in the GCC compiler (it is how the main function is called), as this library contains the printf function. It is also necessary to turn off position independent execution (PIE), to reach the printf() function. This introduces a small security risk (in case of a buffer overflow attack), which is entirely mitigated by the lack of user input. significant amounts of additional work is required to bring the current output up to a good state (such as allowing PIE, making print() not be hardcoded, and implementing an input in a simmilar manner).

Implementation inspired by \cite{LLVMTutorial}