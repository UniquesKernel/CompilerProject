\section{Discussion}
\label{sec:Discussion}

The development of \lang{} and its compiler has largely been a heuristic process
beginning from a state of unfamiliarity with the compilation process and steadily
building the compiler in small increments while applying newly aquired knowledge to
new features. This section is largely dedicated to reflecting on the project, its
results, and the decisions made during development by using the power of hindsight. 

\subsection{The Project}

The Project manages to deliver on most of its objectives, but in hindsight a strong
case can be made that the decision to attempt to build out a compiled programming
language, while also implementing a \borrowChecker{} was too ambitious and that focus
should have been kept on type checking, immutability and new language features. \\

To justify this claim let's take a look at some of the decisions taken during the
project, their impact and why they end up obscuring that the implementation of a
\borrowChecker{} is premature to the projects state.

\subsubsection{Borrow Checking}

Early in the project an attempt to write a custom \lexer{} and \parser{} was
made, which turned out to be very complicated and impeding to the progress of the
overall project. A decision was made to use tools, such as \lexerGen{} and
\parserGen{}, to generate the \lexer{} and the \parser{} from specification files.
This significantly speed of the process of parsing the source code into an \ast{}
which is the fundamental structure for any further work in interpreting or compiling
a programming language. This switch turned out to be very beneficial as language
syntax could now be added very quickly allowing our focus to be turned to code
generation rather than struggling with syntax implementation. \\ 

Next comes a decision to use the visitor pattern to effectively walk the newly
generated \ast{}, this decision turns out to be invaluable, as it modulized the code
to an extend where new features such as the \typeChecker, \borrowChecker{} and
\codeGen{} could be added in a non intrusive way, that didn't require changing the
\ast{} and each visitor implementation was loosely coupled from the the others. \\

Coming back to the heuristic nature of development, after the implementation of the
\codeGen. The next logical step, seems to be to add type checking and borrow checking
to the compiler infastructure, a task that has now been trivialized by the visitor
pattern. While implementing the \borrowChecker{} it becomes apparent quite late in
the development process that many of the required lanaguage features necessary for
the \borrowChecker{} to add value to the language and the compiler - are missing. \\

While the \borrowChecker{} adds a little bit of value to the current state of \lang{}
in hindsight much of the value of the \borrowChecker{} comes from it's ability to
prevent issues when dealing with heap allocated memory and concurrency, not when
dealing with stack allocations as this is automatically cleaned up when exiting any
scope. In hindsight many of the restrictions added to stack based memory comes as a side
effect to the restrictions that the \borrowChecker{} puts on heap allocated memory. 

The time spent on implementing the \borrowChecker{} would have been better
spent implementing language features which could benefit from the use of a
\borrowChecker{} later, as \lang{} matures as a language.

\subsubsection{Type Checking and Immutability}

Unlike the \borrowChecker{} a more nuanced argument can be made for the
implementation of the \typeChecker{} and Immutability in \lang. On one side the
\typeChecker{} much like the \borrowChecker{} detracts from the development of
\lang{} which takes time away from adding much needed features such as the ability
to read and write to standard output and standard input as well as features such as
arrays, strings and heap allocations, which when mature better justify the need for a
\typeChecker{} and a \borrowChecker. \\ 

The argument in favor of the \typeChecker{} is instead that it helps keep the
language stable as it evolves. The language was always intended to be a statically
and strongly typed language and so an argument can be made that evolving the
\typeChecker{} in parallel with each language feature ensures that each new feature
correctly interact with already existing features in the inteded manner. It can be
speculated that type checking is an integral enough part of a language that adding it
afterwards becomes too complicated as the scope of interactions between different
types is more difficult to handle as an after thought. \\ 

\subsubsection{Code Generation and LLVM}

Code generation is a non trivial task, to convert a high level representation like an
\ast{} into assembly or machine level code would require deep knowledge of the targeted
architecture and even more modulization if the code generation needs to be platform
agnostic. The choice to leverage LLVM as a code generation backend is a choice that
aims to keeps the focus on the implementation logic rather than on
the complexity of low-level gode generation. Leading to vastly faster development of
new language features. Most modern compilers today leverage LLVM for this task, with
the exception of the \texttt{Haskell} compiler and even so it lets you swap in the
LLVM code generator backend instead of its own native backend should it be necessary.

\begin{lstlisting}[
  language=c++,
  frame=single,
  numbers=left,
  caption={Code snippet showing how to use llvm to assign values to variables.},
  label={lst:assignment}
  ]
...
llvm::Function *parentFunction = Builder->GetInsertBlock()->getParent();

llvm::AllocaInst *varAllocation = CreateEntryBlockAlloca(
    parentFunction, variableName, getLLVMType(variable->getType()));
...
Builder->CreateAlignedStore(llvm_result, varAllocation, llvm::Align(8));
...
\end{lstlisting}

The code snippet Listing~\ref{lst:assignment} shows in a simplified way how LLVM can
be used to generate code when dealing with the allocation of variables.
Here we use the LLVM \texttt{Builder} to retrieve the function to which the variable
belongs, we create an allocation instruction, with the variable's parent function,
variable name and variable type. Lastly, we store a value in our variable and we can
even specify how the variable should be aligned in memory.\\

The LLVM backend then takes care of making sure that the appropriate code is
generated based on our instructions.

If the project, was writting our own code generator from scratch it would have been
vastly more time consuming to do so, as intimate knowledge is needed about the
targeted architecture and it would have been impossible to write a platform agnostic
code generator capable of generating assembly instructions for a wide range of
architectures. Thus the choice to use LLVM has saved a significant amount of time and
without doubt ensured more reliable code generation. \\

The downside of LLVM is arguably its vast and very complex API which hides away the
process of generating code, which depending on a projects need might not be desirable
but unless a code generator is the core feature of the project this is not so much a
drawback as an advantage. The really issue is that due to the vastness of the API
it's impossible to learn all the features that LLVM has to offer which often means
that as one reads through its documentation, one often stumbles upon features that
immediately antiquates once implementations. \\

An example is they way scopes are handled by the \lang{} compiler which is a very
manual insertion and removal of value in a stack and map based approach. Later in the
development it seems that LLVM itself has a way of implementing a symbol table to
track symbols and scopes. By the time one discovers these new features it's
impossible to swap to using this new feature without rewritting the code generator.

\subsection{The Process}

The development of a programming language and compiler is inherently an iterative
task, with opportunities for continual optimization and improvement. As is typical in
software development, each iteration addresses existing bugs while potentially
introducing new ones. \\

\lang{} was developed using a heuristic MVP (minimum viable progress) process -
developing the compiler and language in small iterations each time adding the
smallest possible feature to solve each objective. \\

Due to the lack of initial domanin knowledge and a need for some structured approach
to development, an initial roadmap was created, with the simplest perceived features
planned for the beginning of the roadmap and more complex features at the end with
the possiblity of adjusting the roadmap as new knowledge was aquired. 

It's arguably the only way to start development, in an area in which one does not have any
expertise or knowledge. Perhaps more research was needed before starting development,
but arguably the better way to gaining knowledge is to begin development and if
necessary leverage newly gained experience to fix past mistakes and to inform future
decisions. \\ 

\newpage
