\subsection{Code Generator}
\label{sec:CodeGenImplement}
When writing the code generator we have taken inspiration from the LLVM tutorial
language Kaleidoscope\cite{LLVMTutorial} to see how to set up boilerplate code and as a
starting point when looking for the correct LLVM functions.

\subsubsection{LLVM}
LLVM works with three important components: a Builder, a Context, and a Module. The
Builder provides an API to create the program instructions, the Context manages the
core global data (types, code blocks and program instructions created by the
builder), while the Module stores the functions.

\paragraph*{Variables}
When the Builder creates a program instruction, it usually returns an \texttt{llvm::Value}
type. It is possible to store these in a \texttt{symbolTable} and access them later,
but because LLVM creates a static single assignment IR, it is not possible to change
the value of the stored variables. To get around this we create an allocation
instance (\texttt{llvm::AllocaInst}), which can hold an \texttt{llvm::value}. We can
then store a new
\texttt{llvm::Value} in the same \texttt{llvm::AllocaInst}, obtaining mutable variables. The exact
implementation is shown in listing \ref{lst:varAssignment}.

\begin{lstlisting}[
  language=c++,
  frame=single,
  numbers=left,
  caption={Code snippet showing how to use llvm to assign values to variables.},
  label={lst:varAssignment}
  ]
//create allocaInst
llvm::AllocaInst *varAllocation = CreateEntryBlockAlloca(
 parentFunction, variableName, getLLVMType(variable->getType()));
  	
//Save variable to symboltable
symbolTableStack.top()[variableName] = varAllocation;

// store value in variable
Builder->CreateAlignedStore(llvm_result, varAllocation, llvm::Align(8));

// Mark if variable is mutable
if (variable->isVarMutable()) {
  mutableVars.top().insert(variableName);
}
\end{lstlisting}

\paragraph*{Functions}

When using the Builder in LLVM to generate function declarations the process of
defining a function in LLVM is similar to how one create subroutines it in assembly
code. Generate first the function using the \texttt{llvm::Function::Create} method and then in
order push your parameters on to the stack, then push all your instructions on to the
stack. When calling the function the same process applies, here we simply collect the
input values and the use \texttt{llvm::CreateCall} to create a function call to whichever
function one intends to call. \\

While LLVM takes care of generating the necesarry object code and GCC uses this to
generate the correct assembly and machine code, it is interesting to note the
similarities in working with LLVM on an abstracted level and working with subroutines on
assembly level to generate the same effects as a high level function call.


\subsubsection{Scope and variables}
To store variables we need a symbol table. The first solution is to create a map from
the variable identifier to the \texttt{llvm::AllocaInst}. This works well as long as
no variables have the same name. This is not good enough, as it should be possible to
reuse variable names as long as it is in different scopes. The symbol table therefore
needs to be expanded to handle scopes.\\ To implement scope in the symbol table, we
changed it, from a map of string identifier to variable allocation, to a stack of
such maps. This way, when we enter a new scope, we can push a new map onto the stack,
thus getting a clean symbol table. When we go out of scope, we the simply pop the
latest map off the stack, removing the variables that were created in the scope. This
also means that it is possible to access variables from higher scopes, as they will
be stored in one of the maps on the stack.\\

To make sure we access the last defined variable with a given identifier, we make a
copy of the stack and check if the variable is in the top map. If not, the top map
is popped off the stack and the next map is searched for the variable. This way the
code generator goes up one scope whenever a variable is not found, and only if the
top level is reached (and the stack is empty) does it conclude that the variable is
missing.

\newpage 

\begin{lstlisting}[
  language=c++,
  numbers=left, 
  frame=single,
  caption={Implementation showing how the code generator search for variables in
  different scopes},
  label={},
  ]
while (!tmpStack.empty()) {
  if (tmpStack.top().find(variable->getVariable()->getName()) !=
      tmpStack.top().end()) {
    loadedVar = tmpStack.top()[variable->getVariable()->getName()];
    if (tmpMutableVars.top().find(variable->getVariable()->getName()) ==
        tmpMutableVars.top().end()) {
      throw std::runtime_error("Reassignment of immutable variable.");
    }
    break;
  }
  tmpStack.pop();
  tmpMutableVars.pop();
}
\end{lstlisting}

To keep track of which variables are mutable, and which are not, the identifiers of
all mutable variables are stored in a list. When a variable is reassigned a value we
check this list, and throw an error if the variable identifier is not found. We have
here the same issue with scope as the symbol table, and the solution is the same: a
stack of lists. When a map is popped of the symbol table we do the same in the
mutable stack.

\subsubsection{Output}
\label{sec:Output}
Implementing an output posed a new type of challenge, as writing an output is not the
responsibility of the compiler, but rather the operating system. As such the
challenge went from finding the correct LLVM method to finding out how to use
external functions in our program. Because an output is not strictly necessary to
answer our problem statement, but more a quality of life, it did become a quick and
unappealing implementation, that needs refinement.\\

The first way to output elements from our  program involved another programming
language (\texttt{C}). The fact that we compile to a object file allowed us to write our
functions, and then link the object file to a \texttt{C} program that would call the \lang{}
functions and print their output with the \texttt{C} implementation of
\texttt{printf}. A workable, but unsatisfying solution, that allowed us to test the
compiler. The next step was to bring \texttt{printf} into \lang. Because our focus
elsewhere, the \texttt{printf} function implementation became a hardcoded reference to
\texttt{printf} with the specific formatting required to print our types.\\

This works only because we already compile with libc linked in the GCC compiler (it
is how the main function is called), as this library contains the \texttt{printf} function.
It is also necessary to turn off position independent execution (PIE), to reach the
\texttt{printf} function. This introduces a small security risk (in case of a buffer
overflow attack), which is entirely mitigated by the lack of user input. Significant
amounts of additional work is required to bring the current output up to a good state
(such as allowing PIE, avoiding hardcoding \texttt{printf} into the compiler, and implementing an input in a similar manner).

