\subsection{Type Checker}
The type checker implements a visitor, as described in section \ref{sec:typeCheckerDesign}. To the types available are \textit{int}, \textit{float}, \textit{char} and \textit{bool}. To keep track of the type of each expression, the base expression (that all expressions inherit from) was given an attribute "type", with public get and set functions. The types where denoted with string types. It was considered using an Enumerator, but in order to get usable error messages from the \lang{} compiler, it was necessary to have the types as string anyway. Having the types as strings also allows for easy expansion of the number of permitted types, as only terminal expressions determines type.\\
Variables are handled with a symbol table that maps the variable identifier to its type. As it is possible to have several variables with the same name if they are in different scopes, the symbol table is actually implemented as a stack of maps, each corresponding to  a single scope. This is explained in more detail in section \ref{sec:CodeGenImplement}, as it is the same technique used in Code Generation.