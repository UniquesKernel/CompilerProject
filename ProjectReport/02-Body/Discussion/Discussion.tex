\section{Discussion}
\label{sec:Discussion}

The development of a programming language and compiler is inherently an iterative
task, with opportunities for continual optimization and improvement. As is typical in
software development, each iteration addresses existing bugs while potentially
introducing new ones. \\

In this context, and considering the project's outcomes in relation to the
pre-established objectives detailed in section~\ref{sec:Objectives}, this section
aims to summarize the current state of the \lang{} compiler, how its current feature
set solves the projects initial objectives and how future features and tools would
help in creating a compiler and language design that further helps solve the initial
objective.

\paragraph*{Current State of \lang{}} \hfill 

The project was initially looking to create design a custom language, which would
have an explicit syntax, type safety and memory safety at compile time. \lang's
current state provides must of the features to support a proto-type language for
solving these issues. \\

It provides all the stable features that one has come to expect for a modern
programming language; variable, control-flow structure such as if-else expressions,
functions and the ability to create loops through recursive functions. \\

The \lang{} compiler enforces to features that ensures some form of explicitness in
\lang{} programs; static typing and immutability. Since \lang{} is statically typed;
all variables and functions are required to have explicit types provided at compile
time - this creates codes in which there can never be any doubt about the type of
data which is being used at any one point in time. The \lang{} compiler also enforces
immutablity by default, leading to a language where data cannot change unexpectedly;
the syntax requires that the intend to mutate data be explicit in the code of any
\lang{} program. These two features help solve the first objective of the project,
besides the construction of a new language and compiler.\\

Borrow checking was supposed to be \lang's attempt at ensuring that the language
would be memory safe despite not having a garbage collector and providing access
system memory through references\footnote{which are a kind of pointer}. It would do
this by ensuring that certain patterns are followed that would ensure that common
pitfalls like null pointers, dangling pointers, double free errors etc. could not
occure thus making the language memory safe without using a garbage collector to
clean up memory.

Unfortunately the \borrowChecker{} still has certain issues which means \lang{} can't
yet deliver on its objective to be memory safe, the unit testing of the
\borrowChecker{} revealed the issues where it some adjustments to the
\borrowChecker's handling and updating of its internal tables might be suffient for
\lang{} to deliver on its memory safety objective.

\paragraph*{Future State and its how it furthers the project objectives} \hfill



\newpage
