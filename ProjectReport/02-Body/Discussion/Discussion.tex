\section{Discussion}
\label{sec:Discussion}

The development of a programming language and compiler is inherently iterative, with
opportunities for continual optimization and improvement. As is typical in software
development, each iteration addresses existing bugs while potentially introducing new
ones. \\

In this context, and considering the project's outcomes in relation to the
pre-established objectives detailed in section~\ref{sec:Objectives}, this section
aims to summarize the achievements of the project. Specifically, it examines the
development of a custom programming language and compiler, designed for type safety
and memory safety without the use of a garbage collector. \\

The \lang{} language has evolved to support common data types such as integers,
floating-point numbers, booleans, and characters. These can be stored in variables
for subsequent use. \lang{} supports essential features like arithmetic operations,
branching, and loop creation via recursive function calls. It does fall short on the
ability to take inputs from users as it is limited only producing outputs. \\

A significant emphasis was placed on explicitness in coding. This was realized
through a static typing system without type inference and by making variables
immutable by default. Mutable states are allowed only through a specialized keyword,
ensuring deliberate intentions for mutability. \\

However, the project encountered challenges in fully implementing the Rust-like
ownership and borrowing rules to ensure memory safety without a garbage collector.
The \borrowChecker{} managed ownership within functions but faced difficulties with
ownership transfer between functions. Consequently, this aspect of the project did
not fully meet the objectives set forth in sections~\ref{sec:Objectives} and
\ref{sec:Requirements}.

\newpage
