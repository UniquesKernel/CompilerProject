%Mathematics
  \usepackage{amssymb}
  \usepackage{amsthm}
  \usepackage{amsmath}
  \usepackage{calculator}
  \usepackage{mathtools}
  \mathtoolsset{showonlyrefs = true}

% Layout and Graphics
  \usepackage{fancyhdr}
  \usepackage{booktabs}
  \usepackage{syntax}
  \usepackage{float}
  \usepackage{titlesec}
  \usepackage{longtable}
  \usepackage{hyperref}
  \usepackage{caption}
  \usepackage{booktabs}
  \usepackage{lastpage}
  \usepackage{verbatim}
  \usepackage{pgf}
  \usepackage{tikz}%, tikz-uml}
  \usepackage{listings}
  \usepackage[
    style=numeric,
    urldate=long
    ]{biblatex}
  \addbibresource{03-EndMatter/References.bib}
  \usepackage[smartEllipses]{markdown}
  \usepackage{url}
  \usepackage
  [
    left = \numberPI cm,
    right = \numberPI cm,
    top = \numberPI cm,
    bottom = \numberPI cm
  ]{geometry}
 
% Tikz Libraries
  \usetikzlibrary{arrows.meta}
  \usetikzlibrary{automata}
  \usetikzlibrary{backgrounds}
  \usetikzlibrary{calc}
  \usetikzlibrary{calendar}
  \usetikzlibrary{decorations}
  \usetikzlibrary{er}
  \usetikzlibrary{folding}
  \usetikzlibrary{intersections}
  \usetikzlibrary{matrix}
  \usetikzlibrary{mindmap}
  \usetikzlibrary{patterns}
  \usetikzlibrary{plothandlers}
  \usetikzlibrary{plotmarks}
  \usetikzlibrary{shapes}
  \usetikzlibrary{trees}
  \usetikzlibrary{positioning}

% Page Layout
  \pagestyle{fancy}
  \fancyhf{}
  \fancyhead[L]{\authorName}
  \fancyfoot[C]{\leftmark}
  \fancyhead[R]{\subject}
  \fancyfoot[R]{\thepage}

  \renewcommand{\footrulewidth}{\numberGOLD pt}
  \renewcommand{\headrulewidth}{\numberPI pt}
  \setlength\parindent{0pt}

% Costume commands
\definecolor{light-gray}{gray}{0.50}
\newcommand{\ansvar}[1]{\footnotesize\vspace{-0.3 cm}{\textbf{\textcolor{light-gray}{Ansvarlig: #1\\}}}\normalsize\\}

\newcommand{\scale}[2]
{
  \MULTIPLY{\numberGOLD}{#1}{\sol}
  #2
}

\setcounter{tocdepth}{4}
\setcounter{secnumdepth}{4}

\newcommand{\N}{\mathcal{N}}
\newcommand{\R}{\mathcal{R}}
\newcommand{\Z}{\mathcal{Z}}
\newcommand{\C}{\mathcal{C}}
\newcommand{\Q}{\mathcal{Q}}
\newcommand{\F}{\mathcal{F}}

\newcommand{\methodDoc}[3]
{
  #1 #2(#3)
}
